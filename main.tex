\documentclass[runningheads, envcountsame, a4paper]{llncs}

\usepackage{amsmath}
\usepackage{amssymb}
\usepackage{graphicx}
\usepackage{makeidx}  
\usepackage{enumerate}
\usepackage{textcomp}         
\usepackage{verbatim}
%\usepackage{subfig}
\usepackage[caption=false]{subfig}

\begin{document}
\mainmatter           
\title{Constructing Parsimonious Hybridization Networks with Multiple Phylogenetic Trees using SAT-solver: Initial Explorations}
\titlerunning{ } 
\toctitle{ }

\author{Vladimir Ulyantsev \and Mikhail Melnik}
\authorrunning{V. Ulyantsev, M. Melnik} 
\tocauthor{Vladimir~Ulyantsev, Mikhail~Melnik}
\institute{ITMO University, Saint-Petersburg, Russia \\
\email{$\{$ulyantsev,melnik$\}$@rain.ifmo.ru}}

\maketitle
\setcounter{footnote}{0}

\begin{abstract}
  TODO: abstract

  \keywords{TODO}
\end{abstract}

\section{Introduction}

Phylogenetic network is a powerful tool to model the reticulate
evolutionary processes (such as horizontal gene transfer and hybrid specification).
Briefly, phylogenetic network is a directed acyclic graph, which has
nodes (called reticulation nodes) with more than one incoming edge. Phylogenetic
networks have been studied by several (?) researchers \cite{huson2010phylogenetic}, \cite{morrison2011introduction}, 
\cite{nakhleh2011evolutionary}. There are several formulations of phylogenetic network
construction problem with various modelling assumptions and different types of input data. 
In this paper we focus on the specific type of phylogenetic network called hybridization
network \cite{semple2006hybridization}, \cite{chen2010hybridnet}.

As an input data for hybridization network construction we consider a set of gene trees. 
All the gene trees have the same set of leaves
Each gene tree models the evolutionary history of some gene. 
Because of the reticulate evolutionary events these trees can have different topologies.
The aim is to construct a hybridization network containing the smallest possible number of 
reticulation nodes and displaying each of the input trees. 

Most of the algorithms for hybridization network construction are heuristic [add PIRN from Wu and some other heuristic algo] 
and they usually deal with only two trees.
However, recently an exact algorithm PIRN has been introduced \cite{wu2013algorithm} which is able to process more than two input trees.

In this paper we introduce a new approach for parsimonious
hybridization network construction from multiply input trees based on satisfiability (SAT) solvers.
%SAT-solver is a software package designed to solve SAT problem, which consists of finding a satisfying assignment for a boolean formula. 
SAT problem is known to be NP-complete \cite{bordewich2007computing}, but state-of-the-art SAT-solvers running on the modern hardware 
are able to solve SAT instances having tens of thousands of variables and hundreds of thousands of clauses in several minutes.

SAT-solver based algorithms have been successfully applied to efficiently calculate evolutionary tree measures \cite{bonet2009efficiently}
as well as to solve problems in other domains such as
finite-state machine induction \cite{heule2010exact}, software verification \cite{}, ...

The general outline of our approach is to convert an instance of hybridization network construction 
problem to an instance of SAT problem (a Boolean formula), solve it using SAT-solver and then convert 
the satisfying assignment into the hybridization network. The formula construction process should ensure that
the network has the minimal possible number of reticulate nodes.

Our approach leads to an exact algorithm outperforming PIRN on ??? test cases out of ???

The paper is structured as follows. Section 2 gives the formal definitions, section 3 describes the Boolean formula
construction process and section 4 gives the experimental results.

\section{Definitions and Background}

We define a "phylogenetic tree" as a leaf-labelled tree constructed over a set of taxa. 
Throughout this paper we assume that trees are rooted and binary, i.e. in-degrees of all nodes, except root, are one, 
and out-degrees are zero for leaves and two for internal nodes. A "hybridisation network" is a directed acyclic graph 
with a single root and leaves bijectively labelled by the set of taxa. If the in-degree of a node is greater than one, 
such node is called hybridisation node. In this paper we assume that hybridisation nodes have in-degree of two and 
out-degree of one. We can do this assumption by noting that we can convert a hybridisation node with in-degree 
of three or more to the sequence of hybridisation nodes of in-degree two \cite{wu2010close}. Other nodes are regular 
tree nodes and have the same properties as in tree.

Consider a hybridisation network $N$. Firstly we keep only one of the incoming edges for each hybridisation node in $N$. 
Secondly we contract edges to remove all the nodes of in-degree one and out-degree one. With this two steps we obtain a 
tree $T'$ from the network $N$.

Now suppose we have a phylogenetic tree $T$ and a hybridisation network $N$. We say that network $N$ \emph{displays} $T$ 
if we can choose the edges of hybridisation nodes in such a way that after edge-contraction obtained tree $T'$ will be 
equivalent to $T$.

A "hybridisation number" is commonly defined as the sum of in-degrees of all edges minus one. Note that under our 
assumptions $h(N)$ simply equals the number of reticulation nodes.

$h(N) = \sum\limits_{v != root} (d^-(v) - 1)$

Suppose we are given a set of $K$ phylogenetic trees $T_1, T_2, ... T_k$ over the same set of taxa. The minimum 
hybridisation network for that set of trees is a network $N_{min}$ that displays each tree and has the lowest 
hybridisation number possible.

The most parsimonious hybridisation network problem.

Given a set of $K$ phylogenetic trees $T_1, T_2, ... T_k$ over the same set of taxa, construct the minimum hybridisation 
network for that set of trees.

It has been shown that even for the case of $K=2$ construction of such a network is NP-complete problem 
\cite {bordewich2007computing}. As far as we know there exists only one algorithm for the construction of the most 
parsimonious hybridisation network \cite {wu2013algorithm}.


\section{Algorithm}

The main idea of the algorithm is to look over possible values if hybridisation number and to construct a boolean 
formula that represent a hybridisation network with fixed hybridisation number. Before doing this search we do some 
pre-processing and consequently some post-processing after the network is found.

\subsection{Pre-processing}

Before the actual boolean formula encoding we modify the input and split it into several tasks to reduce the size 
and complexity of the problem. We do this according to the rules from \cite {bonet2009efficiently}. To define those 
rules we first need to define the term "cluster". A set of taxa $A$ is a cluster in trees $T_1, T_2, ... T_k$ if there 
exists a node in each tree that has all the taxa from $A$ in its descendants.

\textbf{Sub-tree reduction rule}

Replace any sub-tree that appears identically in all input trees by a leaf with new label.

\textbf{Cluster reduction rule}

For each cluster $A$ replace the sub-trees containing it by a leaf with new label and add a new task for processing 
that consists of deleted sub-trees $T'_1, T'_2, ... T'_k$ with leaf set $A$.

After we split the task into a set of less complex tasks, we add a dummy root to each tree of each task along with 
the new dummy leaf for tree consistency. This is done to ensure that all the trees in the input share common root. 
This dummy root will be deleted on the post-processing stage. At this stage we have a set of tasks that will be solved 
separately and then their results will be merged at the post-processing stage.

\subsection{Search of the minimal hybridisation number}

To solve a subtask we need to find the lowest hybridisation number $k$ such that there exists a hybridisation 
network with this hybridisation number. We look through possible values of $k$ starting from the highest (?? why do 
we choose max $k$ equals to the size of taxa ??) and construct a boolean formula corresponding to that $k$. 
We decrease $k$ until the solver can't satisfy the formula, and that means that the previous value of $k$ was the 
lowest possible. There are another strategies of searching the minimal value of $k$. For example we can start from 
zero and increase $k$ until the solver will be able to satisfy the formula, or we can use binary search. But in our 
experiments the most time-consuming computations were with the value of $k_{optimal} - 1$ and consequently results of 
upwards method were very poor. Binary search results were close to downwards method, but in some cases when binary search 
tried to satisfy formulas with values of $k$ less than $k_{optimal}$ its results also were poor. This can be explained by 
an observation that it is much easier to solver to produce an answer if formula is satisfiable than if it is not. In case 
of unsatisfiable formula solver should check every possible answer to ensure that there is no solution. Obvious method 
for reducing searching time is to use different heuristics for finding close starting upper and lower bounds of 
hybridisation number. Possible candidates are PIRN, RIATA-HGT and MURPAR. We do not consider this optimisations in this paper.

\subsection{Encoding boolean formula}

Having a set of trees $T$ over a set of taxa $N$ and fixed hybridisation number $k$ we will construct boolean formula 
that will be satisfiable iff there exists a hybridisation network that displays each tree in $T$ and its hybridisation 
number equals to $k$. To do this we first notice that a network over the set of taxa of size $n$ with hybridisation number 
$k$ will have $2 * (n + k) - 1$ nodes (?? why ??). As we add dummy root and dummy leaf, we finally have $2 * (n + k) + 1$ nodes, 
$k$ of which are hybridisation nodes, $n + 1$ are leaves and others are usual tree nodes. Let $R$ be the set of hybridisation 
nodes, $L$ be the set of leaves and $V$ be the set of regular nodes. We number all the nodes in such a way that all leaves 
have numbers in range $[0,n]$, all regular nodes have numbers in range $[n + 1,2n + k]$ and all hybridisation nodes have numbers 
in range $[2n + k + 1, 2(n + k)]$. And we assume that the number of any leave or regular node is less than number of its parent. 
Such numeration allows us to define the following sets of nodes for each node $v$: $PC(v)$ is the set of possible children of $v$, 
$PP(v)$ is the set of possible parents of $v$ and $PU(v)$ is the set of nodes that can be indirect parents of $v$. Now we will 
describe literals and clauses required to construct boolean formula in conjunctive normal form.

\subsubsection{Network structure encoding}

First of all we encode structure of the network itself. We introduce the following literals: 

\begin{itemize}

\item $\forall v \in V, u \in PC(v) : l_{v,u}(r_{v,u})$ 

$l (r)$ is true iff regular node $v$ has node $u$ as left (right) child.

\item $\forall v \in L \cup V / root, u \in PP(v) : p_{v,u}$ 

$p$ is true iff $u$ is parent of regular node $v$.

\item $\forall v \in R, u \in PP(v) : lp_{v,u}(rp_{v,u})$ 

$lp (rp)$ is true iff $u$ is left (right) parent of hybridisation node $v$.

\item $\forall v \in R, u \in PC(v) : ch_{v,u}$ 

$ch$ is true iff $u$ is child of hybridisation node $v$.

\end{itemize}

This takes $O((n + k)^2)$ literals for specifying network structure, and if we notice that $k < n$ then it is equal 
to $O(n^2)$ literals.

To encode the uniqueness of parents and children we use an obvious pattern that requires $O(n^2)$ clauses. We say 
that the node can have at least one real parent (child or something) and at most one true parent. This can be done 
in the following way (for example for parents of regular node $v$):

$(\bigvee\limits_{u \in PP(v)} p_{v,u}) \wedge (\bigwedge\limits_{i, j \in PP(v);i < j} (p_{v,i} \rightarrow \neg p_{v,j}))$.

We introduce the following constraints to restrict the network structure:

\begin{itemize}

\item at-most-one and at-least-one clauses for literals $l$, $r$, $p$, $lp$, $rp$, $ch$

\item left child must have less number then the right one. This constraint can be translated as 
      
      $\forall u, w \in PC(v): u \geq w$

      $l_{v,u} \rightarrow \neg r_{v,w}$
      
\item same constraint for left and right parents of hybridisation nodes.

\end{itemize}

We should connect parent literals with children literals for network consistency. We do this with the following constraints:

\begin{itemize}

\item if regular node has regular children then children should have this node as parent and vice versa:
    \begin{itemize}
    \item if $u$ is the left child of $v$ then $v$ should be the parent of $u$
    
    $left_{v,u} \rightarrow parent_{u,v}$
    \item the same for the right child
    
    $right_{v,u} \rightarrow parent_{u,v}$ 
    \item if $v$ is the parent of $u$ then $v$ is either its right or left child
    
    $parent_{u,v} \rightarrow (left_{v,u} \vee right_{v,u})$
    \end{itemize}

\end{itemize}

The same idea applies also for other types of parent-child relations.

\begin{itemize}
\item regular parents with hybridisation children
    \begin{itemize}
    \item $left_{v,u} \rightarrow (lp_{u,v} \vee rp_{u,v})$
    \item $right_{v,u} \rightarrow (lp_{u,v} \vee rp_{u,v})$
    \item $lp_{u,v} \rightarrow (left_{v,u} \vee right_{v,u})$
    \item $rp_{u,v} \rightarrow (left_{v,u} \vee right_{v,u})$
    \end{itemize}
    
\item hybridisation parents with regular children
    \begin{itemize}
    \item $ch_{v,u} \rightarrow parent_{u,v}$
    \item $parent_{u,v} \rightarrow ch_{v,u}$
    \end{itemize}
    
\item hybridisation parents with hybridisation children
    \begin{itemize}
    \item $ch_{v,u} \rightarrow (lp_{u,v} \vee rp_{u,v})$
    \item $lp_{u,v} \rightarrow ch_{v,u}$
    \item $rp_{u,v} \rightarrow ch_{v,u}$
    \end{itemize}
    
\item we also should order parents and children of hybridisation nodes, for consistent numeration

	$\forall v, u, w: v \in R, u \in PC(v), w \in PP(v): w \leq u$
    \begin{itemize}
    \item $ch_{v,u} \rightarrow \neg lp_{v,w}$
    \item $ch_{v,u} \rightarrow \neg rp_{v,w}$
    \end{itemize}

\end{itemize}

\subsubsection{Mapping trees to network}

To show that network should contain all the input trees we add literals that represent mapping the tree over 
the network. We should note that leaves of the trees are bijectively mapped to leaves of the network. 
(?? The intuition for these literals is that ??):

\begin{itemize}

\item $\forall t \in T, s \in V(t), v \in V : x_{t,s,v}$

$x$ is true iff regular node $v$ represents node $s$ from tree $t$, i.e. $x$ literals represent bijective 
mapping of network nodes to tree nodes

\item $\forall t \in T, v \in R : dir_{t,v}$

$dir$ is true iff left parent edge of hybridisation node $v$ is used to display tree $t$ and it is false in case 
of right edge, i.e. it specifies direction of necessary parent to display current tree

\item $\forall t \in T, v \in R : rused_{t,v}$

$rused$ is true iff child of hybridisation node $v$ is used to display tree $t$

\item $\forall t \in T, v \in V : used_{t,v}$

$used$ is true iff regular node $v$ is used to display tree $t$

\item $\forall t \in T, v \in V, u \in PU(v) : up_{t,v,u}$

$up$ is true iff regular node $u$ is indirect parent of node $v$ in the network but it is its direct parent in tree $t$

\end{itemize}

We use $up$ variables because edge contraction is used for displaying trees, so it is not enough to only 
consider direct parents in network.

Now we define constraints that show whether network node is used to display an input tree:

\begin{itemize}

\item First of all we forbid meaningless values of $used$ literals.

	If the direction of parent of hybridisation node for specific tree is defined, we shouldn't use another branch:
    \begin{itemize}
    \item $\forall t \in T, v \in R, u \in PP(v):$
    
    $(\neg dir_{t,v} \wedge lp_{v,u}) \rightarrow \neg used_{t,u}$
    
    $(dir_{t,v} \wedge rp_{v,u}) \rightarrow \neg used_{t,u}$
    \end{itemize}
	Also if we don't use current hybridisation node for current tree, we shouldn't use its parents for current tree:
	\begin{itemize}
	\item $\forall t \in T, v \in R, u \in PP(v):$
	
	$(lp_{v,u} \wedge \neg rused_{t,v}) \rightarrow \neg used_{t,u}$
	
	$(rp_{v,u} \wedge \neg rused_{t,v}) \rightarrow \neg used_{t,u}$
	\end{itemize}

\item Secondly we limit usage of hybridisation nodes.

	$\forall t \in T, v \in R, u \in PP(v):$
	
    If $u \in V \cup L$, i.e. $u$ is regular tree node. Then we have no choice but to use hybridisation node:

    \begin{itemize}
	\item $ch_{t,v,u} \rightarrow rused_{t,v}$
	\end{itemize}
    
    Then consider $u \in R$.
    
    If child is not used for current tree then we shouldn't use the node itself:

    \begin{itemize}
    \item $\neg rused_{t,u} \rightarrow \neg rused_{t,v}$
    \end{itemize}

	We also should forbid usage of current node when its direction differs from direction specified in its child, 
	and we should use it if its direction matches direction specified in its child and its child itself is used:

    \begin{itemize}
    \item $(lp_{t,u,v} \wedge \neg dir_{t,u}) \rightarrow \neg rused_{t,v}$
    \item $(rp_{t,u,v} \wedge dir_{t,u}) \rightarrow \neg rused_{t,v}$
    \item $(lp_{t,u,v} \wedge dir_{t,u} \wedge rused_{t,u}) \rightarrow rused_{t,v}$
    \item $(rp_{t,u,v} \wedge \neg dir_{t,u} \wedge rused_{t,u}) \rightarrow rused_{t,v}$
    \end{itemize}
    
\item At-least-one and at-most one clauses for literals $x$ and $up$. Note that in case of $x$ literals we have 
restriction that at most one node from tree corresponds to the network node, and that at most one node from network 
corresponds to the tree node.

\item If we know that node from tree bijectively matches some network node then we should use that node for that tree:
    \begin{itemize}
	\item $x_{t,s,v} \rightarrow used_{t,s}$
    \end{itemize}
    Note that due to dummy roots we know that all tree roots match network root:
    \begin{itemize}
	\item $x_{t,s,root}$
    \end{itemize}

\item Constraints for regular node $v$ and its relation to direct tree parent from tree $t$.

	If parent $u$ of node $v$ is used in tree $t$, then it is its direct parent in corresponding tree and vice versa:

    \begin{itemize}
    \item $\forall t \in T,v \in V \cup L, u \in PP(v), u \in V$
    
    $(parent_{v,u} \wedge used_{t,u}) \rightarrow up_{t,v,u}$
    
    $(parent_{v,u} \wedge up_{t,v,u}) \rightarrow used_{t,u}$
	\end {itemize}
	
	If we do not use parent $u$ of node $v$ in tree $t$, we should traverse direct tree parent from $u$ to $v$ and vice versa:

	\begin{itemize}
    \item $\forall t \in T,v \in V \cup L, u \in PP(v), u \in V, w \in PP(u)$
    
	$(parent_{v,u} \wedge \neg used_{t,u} \wedge up_{t,u,w}) \rightarrow up_{t,v,w}$
    
    $(parent_{v,u} \wedge \neg used_{t,u} \wedge up_{t,v,w}) \rightarrow up_{t,u,w}$
	\end {itemize}
	
	If parent $u$ of node $v$ is hybridisation node, we should care about order of $v$ and parents of $u$:

	\begin{itemize}
    \item $\forall t \in T,v \in V \cup L, u \in PP(v), u \in R, w \in PU(u)$

	if $w \leq v$ then $w$ can not be parent of $u$ considering tree $t$:
	
	$parent_{v,u} \rightarrow \neg up_{t,u,w}$
	
	otherwise we should traverse direct tree parents from $u$ to $v$ (?? and maybe vice versa ??):
	
	$(parent_{v,u} \wedge up_{t,u,w}) \rightarrow \neg up_{t,v,w}$
    \end{itemize}

\item Constraints for hybridisation node $v$ and its relation to direct tree parent from $t$.

	If direction of parent $u$ matches direction specified in node $v$ and parent is used in $t$ then is is direct parent of $v$ in $t$:

    \begin{itemize}
    \item $\forall t \in T, v \in R, u \in PP(v), u \in V$
	
	$(lp_{v,u} \wedge dir_{t,v} \wedge used_{t,u}) \rightarrow up_{t,v,u}$
    
    $(rp_{v,u} \wedge \neg dir_{t,v} \wedge used_{t,u}) \rightarrow up_{t,v,u}$
	\end {itemize}
	
	If parent $u$ us not used in tree $t$ then we should traverse direct tree parents from $u$ to $v$ ( ?? and maybe vice versa ??):

    \begin{itemize}
    \item $\forall t \in T, v \in R, u \in PP(v), u \in V, w \in PU(u)$
	
    $(lp_{v,u} \wedge dir_{t,v} \wedge \neg used_{t,u} \wedge up_{t,u,w}) \rightarrow up_{t,v,w}$

    $(rp_{v,u} \wedge \neg dir_{t,v} \wedge \neg used_{t,u} \wedge up_{t,u,w}) \rightarrow up_{t,v,w}$
	\end {itemize}
	
	If parent $u$ is a hybridisation node we also should traverse direct tree parents from $u$ to $v$ ( ?? and maybe vice versa ??):
	
    \begin{itemize}
    \item $\forall t \in T, v \in R, u \in PP(v), u \in R, w \in PU(u)$
	
    $(lp_{v,u} \wedge dir_{t,v} \wedge up_{t,u,w}) \rightarrow up_{t,v,w}$

    $(rp_{v,u} \wedge \neg dir_{t,v} \wedge up_{t,u,w}) \rightarrow up_{t,v,w}$
	\end {itemize}

\item Consider restrictions on $x$ literals.

	If $v$ is leaf and its parent $u$ bijectively maps to node $tu$ in tree $t$ then it is direct parent.	    

	\begin{itemize}
    \item $\forall t \in T, v \in L, u \in PP(v), tu \in V(t)$

	$x_{t,tu,u} \rightarrow up_{t,v,u}$
	\end {itemize}
	
	If node $v$ and its parent $u$ both bijectively map to child and parent in tree $t$, then $u$ is direct 
	parent of $v$ considering tree $t$.

	\begin{itemize}
    \item $\forall t \in T, v \in V, u \in PP(v), tv \in V(t), tu = p(tv)$

	$(x_{t,tv,v} \wedge x_{t,tu,u}) \rightarrow up_{t,v,u}$
	\end {itemize}
	
	On the other hand if we know that $u$ is direct parent of $v$ in tree $t$, then we also have bijective mapping for them.

	\begin{itemize}
    \item $\forall t \in T, v \in V, u \in PP(v), tv \in V(t), tu = p(tv)$

	$(x_{t,tv,v} \wedge up_{t,v,u}) \rightarrow x_{t,tu,u}$
	\end {itemize}
	
	We also should care about numeration. Thus we can not map node $u$ from network to node $tu$ from tree if 
	number of $u$ is less than number of its child $v$.

	\begin{itemize}
    \item $\forall t \in T, v \in V, u \in PP(v), tv \in V(t), tu = p(tv), u < v$

    $x_{t,tv,v} \rightarrow \neg x_{t,tu,u}$
	\end {itemize}
	
\end {itemize}

\subsubsection{Heuristics}

\begin{itemize}
	\item We add some heuristic constraints by noticing an obvious fact that number of node in network can not be 
	less than the size of its sub-tree and it can not be greater than $|t| - depth(tv)$.

	\begin{itemize}
    \item $\forall t \in T, v \in V, tv \in V(t), tv < |sub-tree(tv)|$

    $\neg x_{t,tv,v}$
    
    \item $\forall t \in T, v \in V, tv \in V(t), tv > |t - depth(tv)|$
    
    $\neg x_{t,tv,v}$
	\end {itemize}
	
	\item Consider node $tv_1$ in tree $t_1$ and node $tv_2$ in tree $t_2$ such that their sub-trees have disjoint 
	set of taxa. Then these nodes can not be mapped to the same node in network.
	
	\begin{itemize}
    \item $\forall t_1, t_2 \in T, v \in V, tv_1 \in V(t_1), tv_2 \in V(t_2)$

    $\neg x_{t_1,tv_1,v} \vee \neg x_{t_2,tv_2,v}$

	\end {itemize}

\end {itemize}

\subsection{Solving boolean formula}

To solve generated boolean formula we use a SAT-solver CryptoMiniSat 4.2.0. Choosing the most appropriate solver 
is not considered in current paper, however several experimentations were made by M. Bonet and K. John \cite {bonet2009efficiently} 
so it is one of the topics of further research.

\subsection{Post-processing}

After solving a task we recover network from the SAT-solver output. After that we delete dummy root and appropriate 
leaf from the network. And when all the subtasks of the original task are solved, we merge their networks into one 
hybridisation network corresponding to original task.

\section{Results}

To test performance of our algorithm we evaluated it on a grass (Poaceae) dataset provided by the Grass Phylogeny Working Group 
(Grass Phylogeny Working Group, 2001). We compared results our method with results of PIRN in exact and in heuristic modes. 

\section{Discussion}

Testing shows that our method outperform exact PIRN algorithm in all cases.

\section*{Acknowledgements}

\bibliographystyle{splncs}
\bibliography{main}
\clearpage
\end{document}
